\documentclass{tufte-handout}

\title{From Idea to Publication: managing projects for academic work}

\author[lmp]{Luke M Perez\thanks{https://github.com/lukemperez/Project-Mgt-Short-Course}, University of Texas at Austin}

%\date{28 March 2010} % without \date command, current date is supplied

% \geometry{showframe} % display margins for debugging page layout

\usepackage{graphicx} % allow embedded images
  \setkeys{Gin}{width=\linewidth,totalheight=\textheight,keepaspectratio}
  \graphicspath{{graphics/}} % set of paths to search for images
\usepackage{amsmath}  % extended mathematics
\usepackage{booktabs} % book-quality tables
\usepackage{units}    % non-stacked fractions and better unit spacing
\usepackage{multicol} % multiple column layout facilities
\usepackage{lipsum}   % filler text
\usepackage{fancyvrb} % extended verbatim environments
  \fvset{fontsize=\normalsize}% default font size for fancy-verbatim environments

% Standardize command font styles and environments
\newcommand{\doccmd}[1]{\texttt{\textbackslash#1}}% command name -- adds backslash automatically
\newcommand{\docopt}[1]{\ensuremath{\langle}\textrm{\textit{#1}}\ensuremath{\rangle}}% optional command argument
\newcommand{\docarg}[1]{\textrm{\textit{#1}}}% (required) command argument
\newcommand{\docenv}[1]{\textsf{#1}}% environment name
\newcommand{\docpkg}[1]{\texttt{#1}}% package name
\newcommand{\doccls}[1]{\texttt{#1}}% document class name
\newcommand{\docclsopt}[1]{\texttt{#1}}% document class option name
\newenvironment{docspec}{\begin{quote}\noindent}{\end{quote}}% command specification environment

% Bibliography 
% \usepackage[english]{babel}
% \usepackage[authordate,bibencoding=auto,strict,backend=bibtex,natbib]{biblatex}
% \addbibresource{Proj_MGT.bib}


\begin{document}



\maketitle% this prints the handout title, author, and date

% \begin{abstract}
% \noindent
% Project management is fundamentally about managing the anxiety
% produced by ``open loops'' and unfinished tasks. This handout surveys
% a few strategies which help close the open loops and increase
% productivity. Your mileage may vary.
% \end{abstract}

%\printclassoptions

\section{Introduction} % (fold)
\label{sec:intro}

At its most basic, project management not much more than axiety management. Every project invariably has ``open loops'' of unfinished tasks which the mind attempts to remind itself to do.\marginnote{The mind is awful at reminding you what to when you need to be reminded. Use a task management system that you trust.} A reliable task management \textit{system}---which is not the same thing as an application or program---helps you focus on the work by closing the loops and reducing your anxiety about forgotten tasks.

% \subsection{Close the loops}\marginnote{David Allen \textit{Getting Things Done}} % (fold)
% \label{sub:close_the_loops}
% Managing projects is, at its core, fundamentally about managing the
% anxiety produced by the nagging question in your head ``Oh, don't
% forget to do [x].'' As it turns out, human nature is surprisingly
% inefficient when it comes to remembering tasks, ``to-do'' lists, and
% priorities.
% % subsection close_the_loops (end)

% \subsection{Apps don't matter} % (fold)
% \label{sub:apps_don_t_matter}

% You can waste a lot of hours jumping from application to application,
% pursuing task-management utopia. But remember that at the end of the
% day, what application you use does not matter. Find the ones that work
% for you, learn the hell of it.\marginnote{Kieran Healy, \textit{Plain
%     Text Guide}} Grasping a few basic concepts for \textit{process}
% will serve you much better. Let the tasks determine the tools, not the
% other way around.
% % subsection apps_don_t_matter (end)
% % section introduction (end)

\subsection{A GTD Crash Course}
\label{gtd-crash-course}

\paragraph{Projects vs Tasks}
\label{sec:proj-tasks}

\begin{enumerate}
\item[] A task is any single action.
\item[] A project is anything that requires two or more actions
\end{enumerate}

\paragraph{A simple example}

Emailing your adviser to ask a question about the grading rubric for a
TA class is a task because it is one action. Attending office hours
for your adviser is a project because you have several dependencies
that are involved: drafting a quick agenda of the two or three items
you need. Emailing him or her the agenda in advance with any
attachments they need so they can be prepared. Spending a few minutes
before you arrive to review your agenda before walking to his or her
office, etc., etc. 


\subsection{Tasks vs Contexts}
\label{tasks-contexts}

Every task you create for yourself belongs to both a project and a
context.\sidenote{Sometimes, the project may just be a catch all for
  all single action tasks.} A \textit{Context} is often described as
the one physical thing or place that is mandatory before the task can
be done. For instance, you need your laptop if you're going to write a
section of your paper but you must be in the building if you're going to drop
off documents to anyone in the front office.\marginnote{Pro-tip: Make
  a context or just a list for every faculty or staff member you
  interact with regularly. When you get two or three items, you have
  an agenda to email them. It let's them know you value their time.} Simple enough. But you
can extend this concept to non-physical things like the day and time
of the week or your energy level. The reason for doing this is that
you can maximize you're overall efficiency through the day.
% subsection Tasks-Contexts (end)

\subsection{Daily and Weekly Review}
\label{sec:review}

Whatever system you use to manage your projects, it will not work if
you do not trust it intuitively. The instant you're not sure if you've
captured all your tasks and projects, you're mind will begin to
``remind'' you---at which point, the anxiety of unfinished work and
open loops will set in. The only way to prevent this is to review
regularly. Spending a few minutes at the start and end of the work day
to review will save you countless hours of time and anxiety. A longer
weekly review helps you keep your eye on the medium- and
long-term.\footnote{David Allen discusses this process in greater
  detail.}


% subsection Review (end)

\section{Backward Planning} % (fold)
\label{sec:backward_planning}

Most graduate students think about projects as a linear move from data, to theory, to write up, to submission. This is wrong for two reasons. First, the process is literally backwards. Project planning should begin with a deadline and worked-out in reverse chronological order based on the required dependencies and their estimated durations. Second, nearly every piece of writing we create develops from a previous paper or idea. Writing and the creative process are iterative. The creative process is a necessary prerequisite to the writing process, but they are not the same thing. You need a way to keep tabs on new ideas, develop them, and recognize when they are ready to be made into a project.

% The approach that most graduate students think about projects is
% progressively linear. From the moment we get an idea or decide to
% write a paper, the process usually looks something like the following.

% \[ data~ crunch \rightarrow read~ for~ a~ theory \rightarrow recrunch~
%   data \rightarrow repeat~ until~ deadline \]
% % \[ Idea \righarrow Read \rightarrow Find data? \rightarrow read more \rightarrow cram for the deadline \]

% This process is \textit{literally} backwards. Although somewhat of a
% caricature of the writing process, some version of this is where most
% of us find ourselves most of the time. Parkinson's Law captures the
% phenomenon but not the reason why this happens on an individual
% level. Open loops cause us to constantly revise, re-read, refine
% because we're unsure when the project is 'done'.\marginnote{Pro-Tip:
%   it is always a good idea to set your deadline a week before the hard
% deadline.} Better is to begin
% with the hard deadline and work in reverse-chronological order to
% determine all the dependencies which must be accomplished, the time
% estimated for each dependency, and the order they must be
% finished. The process looks like this:

% \[Read \leftarrow Data~ Crunch \leftarrow Write \leftarrow Proof  \leftarrow Revisions \leftarrow Deadline  \]

% \noindent For example, if you need a week to proof read something, then
% you need to be finished at least one week before you deadline. And if
% you write the introduction and conclusions last and those take a few
% days, then you need to finish the data-analysis or major portions at
% least a week before that. And so forth...\marginnote{Although a bit of
%   overkill for graduate students, a Gantt chart captures this concept
%   quite well.}
% section backward_planning (end)

% \section{In the weeds}
% \label{in-weeds}


% section In the Weeds (end)

\section{Writing Projects and GTD}
\label{sec:write-proj-gtd}

\subsection{Basic Principles}
\label{basic-principles}

\begin{enumerate}

  \item Embrace iteration
  \item Write early, write often
  \item Organize efficiently
  \item Outsource your time-management

  \end{enumerate}

\paragraph{Embrace Iteration} No project goes from idea to publication
in one move. Keep a notepad or journal on you at all time. Find a good database program to elaborate the ideas. Review your ideas regularly. Tinker, draft, trash, repeat. Lots of scholars keep a spreadsheet with columns for \textsc{Title, Thesis, Abstract}. I use a database system. Again, find what works for you, and make it a habit. 

\paragraph{Write early, Write often} Get in the habit of writing 300--500 words every day, or at a minimum several times per week. The research
shows that scholars who write for short periods more often produce
more work, more frequently than those who write for longer periods in
blocks.\footnote{Paul J. Silva, \textit{How to write a lot.}} One
  reason this works is that it trains yourself to overcome writer's
  block and anxiety about getting started.\footnote{Steven
    Pressfield calls this ``killing the resistance.''}

\paragraph{Organize efficiently}

Good ideas are only useful if you can find them later. Using a good file naming system goes a long way to keep things tidy on your hard drive. Writing
notes in plain text with a file name that begins with \texttt{YYYYMMDD
  -- title} will help you find your notes faster whether you use a
program like Evernote, DevonThink, or manual
searching.\marginnote{Think of your notes as digital versions of index
  cards. cf. Umberto Eco, \textit{How to Write a Thesis}}. 

% I use a database program called \textit{DevonThink} which is best
% described as much more advanced version of Evernote. It has an
% indexing feature and built-in AI that can find related notes. 

\paragraph{Outsource your time-management}

Using a task management system with the concepts outlined above is, I
believe, the key hurdle to getting work out the door. Just remember that it isn't about the application or which brand journal you buy---but it is essential to (a) find a system that works for you, and (b) use that system until it's second nature.\marginnote{Pro-Tip: Don't try do learn all this stuff at once. Try to master one habit per semester.}

% \footnote{Seth Godin's talk on ``the lizard brain'' makes the same point while discussing product shipment to app developers.} Whatever application you use, make it a habit so that you close your loops and trust it. 


%   I've been using
% \textit{OmniFocus}, which is a Mac-only product. The key feature is
% that it has companion apps for iOS which lets me stay on top of my
% tasks even when I don't have my computer at hand. There are other
% apps of course. If you have one, keep using it until it doesn't work
% for you. But you could easily do this with a notepad, calendar, and a
% pencil.  

\section{Applications} % (fold)
\label{sec:applications}

\paragraph{Currently Using} % (fold)
\label{par:currently_using}
\begin{enumerate}
  \item[] Leuchtturm 1917 Journal and a good pen
  \item[] DevonThink
  \item[] Emacs \& Sublime
  \item[] OmniFocus for Mac and iOS
  \item[] BusyCal
\end{enumerate}
% paragraph currently_using (end)

\paragraph{Previously Used} % (fold)
\label{par:previously_used}

\begin{enumerate}
  \item[] Evernote
  \item[] Scrivener\footnote{Scriverner is strongly encouraged for those who are not comfortable with the command line}
  \item[] Mellel (Word processor)
  \item[] Sente and/or Pages (the citation manager)
\end{enumerate}

% paragraph previously_used (end)

\paragraph{Noteable Mention} % (fold)
\label{par:noteable_mention}

\begin{enumerate}
  \item[] TexPad
  \item[] Mendeley
  % TextExpander
  % LaunchBar
\end{enumerate}
% paragraph noteable_mention (end)
% section applications (end)
\nocite{*}

% \printbibliography[title=Futher Reading]
\renewcommand{\bibname}{Further Reading} 
\bibliography{Proj_MGT}
\bibliographystyle{plainnat}

\end{document}
